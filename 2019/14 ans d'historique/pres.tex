\documentclass{beamer}


\title{Brancher Anyblok à 14 ans d'historique métier}
\subtitle{Une sombre histoire de PHP, MySQL et MsSQL}
\author{Jean-Sébastien SUZANNE et Hugo QUEZADA}

\newcommand{\TwitterLogo}{\protect\includegraphics[height=
1.7ex,keepaspectratio]{./twitter.png}}
\newcommand{\EmailLogo}{\protect\includegraphics[height=
1.7ex,keepaspectratio]{./email.png}}
\newcommand{\AnyBlokLogo}{\protect\includegraphics[height=
1.7ex,keepaspectratio]{./anyblok.png}}

\usetheme{Amsterdam}
\begin{document}
	\frame {
		\titlepage
	}
	
	\frame {
		\frametitle{Qui sommes nous ?}
		
		\begin{columns}
            \begin{column}{3cm}
                Logo Sensee
			\end{column}
			\begin{column}{3cm}
				Logo LMC
			\end{column}
		\end{columns}
			
		\hfill \break			
		\begin{columns}
            \begin{column}{5cm}
                \textbf{\AnyBlokLogo{} Sébastien Suzanne}
		    	\begin{itemize}
			  		\item Connu aussi sous le speudo de PAPABLOK
			  		\item \TwitterLogo{} @jssuzanne
			    	\item \EmailLogo{} js.suzanne@sensee.com
		    	\end{itemize}
			\end{column}
			\begin{column}{5cm}
				\textbf{Hugo Quezada}
		    	\begin{itemize}
			  		\item Dis le petit Basque du Chili
			  		\item \TwitterLogo{} @jssuzanne
			    	\item \EmailLogo{} h.quezada@sensee.com
		    	\end{itemize}
			\end{column}
		\end{columns}
	}
	
	\frame{
		\frametitle{L'éxistant}
		\framesubtitle{14 ans de code...}
		\begin{itemize}
			\item Code légacy en PHP5 (pas de troll SVP) 
			\item  Écosytème comportant plusieurs Sgbd sans API (MySQL, MsSQL)
			\item  Schémas de bdd multiples
			\item  Pas de tests
			\item  Pas de CI
			\item  Beaucoup de développeurs et peu d'historique écrit
			\item  pas d'ORM 
		\end{itemize}
	}
	
	\frame{
		\frametitle{Projet éxistant}
		\framesubtitle{Une architecture découpé}
		\setlength{\unitlength}{1mm}
		\begin{picture}(0,0)
			\put(40,20){\framebox(20,10){Front}}
			\put(40, 0){\framebox(20,10){MySQL}}
			\put(50,20){\line(0,-1){10}}
			\put(00, -20){\framebox(20,10){Backend}}
			\put(50, 00){\line(-2,-1){30}}
			\put(80,-20){\framebox(40,10){Tâche asynchrones}}
			\put(50, 00){\line(2,-1){30}}
			\put(0,-40){\framebox(20,10){MsSQL :( }}
			\put(10, -20){\line(0,-1){10}}
		\end{picture}
	}	
	\frame{
		\frametitle{Vision final du projet}
		\framesubtitle{Objectif}
		\begin{itemize}
		    \item Un code python propre et testé
			\item Une API simple et uniforme, utilisé dans tous nos projets
			\item Une application plus proche des standards actuels
			\item Un projet plus séduisant et plus attrayant pour des potentiels futurs développeurs 
		\end{itemize}
	}
	\frame{
		\frametitle{Vision final du projet}
		\framesubtitle{Stratégie}
		\begin{itemize}
		    \item reprise des tables en étant plus explicite sur les noms des colonnes avec AnyBlok
			\item TDD
			\item Strangler pattern
		\end{itemize}
	}
	\frame{
		\frametitle{Stratégie}
		\framesubtitle{Strangler pattern}
		
		Définition du pattern
		
		\begin{itemize}
			\item Diminue le risque 
			\item petite équipe (de 1)
			\item permet de faire cohabiter les 2 versions
		\end{itemize}
	}
	\frame{
		\frametitle{Stratégie}
		\framesubtitle{Anyblok: Présentation}
	}
	\frame{
		\frametitle{Stratégie}
		\framesubtitle{Anyblok: Écrire des tests pour notre Model}
	}
	\frame{
		\frametitle{Stratégie}
		\framesubtitle{Anyblok: Définir le Model}
	}
	\frame{
		\frametitle{Stratégie}
		\framesubtitle{Anyblok: Définir Model sur une table existante}
	}
		
\end{document}
